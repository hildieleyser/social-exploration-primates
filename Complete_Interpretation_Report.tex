\documentclass[11pt]{article}
\usepackage[utf8]{inputenc}
\usepackage{geometry}
\usepackage{amsmath}
\usepackage{amsfonts}
\usepackage{amssymb}
\usepackage{graphicx}
\usepackage{booktabs}
\usepackage{longtable}
\usepackage{array}
\usepackage{multirow}
\usepackage{wrapfig}
\usepackage{float}
\usepackage{colortbl}
\usepackage{pdflscape}
\usepackage{tabu}
\usepackage{threeparttable}
\usepackage{threeparttablex}
\usepackage{makecell}
\usepackage{xcolor}
\usepackage{setspace}
\usepackage{hyperref}

\geometry{margin=1in}
\setlength{\parindent}{0pt}
\setlength{\parskip}{6pt}

\title{\textbf{Hierarchical Multinomial Analysis: Complete Interpretation Guide}}
\author{Analysis Team}
\date{\today}

\begin{document}

\maketitle

\section{Executive Summary}

This report provides a comprehensive interpretation guide for the hierarchical multinomial Bayesian regression analysis of primate social decision-making. The analysis examines 1,474 experimental trials from 6 rhesus macaques across three social contexts (solo, duo, trio) with three possible outcomes: exploit, explore, and none.

\section{Model Overview}

\subsection{Research Question}
How do social context, individual differences, and value-based factors influence primate decision-making in explore-exploit scenarios?

\subsection{Model Structure}
The hierarchical multinomial logistic regression model predicts three outcomes:
\begin{itemize}
    \item \textbf{Exploit:} Choose the known, safe option
    \item \textbf{Explore:} Choose the uncertain, potentially rewarding option
    \item \textbf{None:} Choose not to participate
\end{itemize}

\section{Statistical Outputs: Interpretation Guide}

\subsection{Model Comparison Table}

\begin{table}[h]
\centering
\begin{tabular}{lcccc}
\toprule
\textbf{Model} & \textbf{AIC} & \textbf{BIC} & \textbf{Parameters} & \textbf{$\Delta$AIC} \\
\midrule
Null & 3,242.7 & 3,253.3 & 2 & 428.7 \\
Fixed Effects & 3,031.7 & 3,084.7 & 8 & 217.7 \\
\textbf{Hierarchical} & \textbf{2,814.0} & \textbf{2,909.3} & \textbf{18} & \textbf{0.0} \\
\bottomrule
\end{tabular}
\caption{Model comparison results}
\end{table}

\textbf{How to interpret:}
\begin{itemize}
    \item \textbf{AIC (Akaike Information Criterion):} Lower values indicate better fit. The hierarchical model has the lowest AIC (2,814.0), indicating it provides the best balance of fit and parsimony.
    \item \textbf{BIC (Bayesian Information Criterion):} Similar to AIC but penalizes complexity more heavily. The hierarchical model also has the lowest BIC.
    \item \textbf{$\Delta$AIC:} Difference from the best model. Values > 10 indicate substantial differences in model quality.
    \item \textbf{Parameters:} Number of estimated parameters. More parameters allow better fit but risk overfitting.
\end{itemize}

\subsection{Coefficient Table}

\begin{table}[h]
\centering
\begin{tabular}{lcccc}
\toprule
\textbf{Parameter} & \textbf{Estimate} & \textbf{SE} & \textbf{Z-value} & \textbf{P-value} \\
\midrule
\multicolumn{5}{l}{\textbf{Explore vs Exploit}} \\
\hline
Intercept & 0.241 & 0.194 & 1.25 & 0.212 \\
Social Complexity & -0.054 & 0.095 & -0.56 & 0.573 \\
Expected Explore & 0.290 & 0.072 & 4.01 & <0.001 \\
Subjective Exploit & -0.525 & 0.068 & -7.67 & <0.001 \\
Rank & 0.055 & 0.102 & 0.54 & 0.590 \\
\hline
\multicolumn{5}{l}{\textbf{None vs Exploit}} \\
\hline
Intercept & -1.482 & 0.230 & -6.45 & <0.001 \\
Social Complexity & 0.845 & 0.105 & 8.04 & <0.001 \\
Expected Explore & -0.020 & 0.076 & -0.26 & 0.794 \\
Subjective Exploit & -0.553 & 0.074 & -7.48 & <0.001 \\
Rank & 0.901 & 0.101 & 8.90 & <0.001 \\
\bottomrule
\end{tabular}
\caption{Model coefficients with significance tests}
\end{table}

\textbf{How to interpret coefficients:}
\begin{itemize}
    \item \textbf{Estimate:} Log-odds ratio. Positive values increase the probability of that outcome relative to the reference (Exploit).
    \item \textbf{SE (Standard Error):} Uncertainty in the estimate. Smaller values indicate more precise estimates.
    \item \textbf{Z-value:} Test statistic. Values > 2 or < -2 indicate statistical significance.
    \item \textbf{P-value:} Probability of observing this result by chance. P < 0.05 indicates statistical significance.
\end{itemize}

\textbf{Key findings:}
\begin{itemize}
    \item \textbf{Social Complexity:} Strongly increases None responses (Z = 8.04, p < 0.001)
    \item \textbf{Expected Explore:} Strongly increases Explore responses (Z = 4.01, p < 0.001)
    \item \textbf{Subjective Exploit:} Decreases both Explore and None responses (Z = -7.67, -7.48)
    \item \textbf{Rank:} Strongly increases None responses (Z = 8.90, p < 0.001)
\end{itemize}

\subsection{Odds Ratios}

\begin{table}[h]
\centering
\begin{tabular}{lcc}
\toprule
\textbf{Effect} & \textbf{Explore vs Exploit} & \textbf{None vs Exploit} \\
\midrule
Social Complexity & 0.947 & 2.327 \\
Expected Explore & 1.336 & 0.980 \\
Subjective Exploit & 0.592 & 0.575 \\
Rank & 1.057 & 2.461 \\
\bottomrule
\end{tabular}
\caption{Odds ratios (exponentiated coefficients)}
\end{table}

\textbf{How to interpret odds ratios:}
\begin{itemize}
    \item \textbf{Values > 1:} Increase the probability of that outcome
    \item \textbf{Values < 1:} Decrease the probability of that outcome
    \item \textbf{Example:} Social Complexity OR = 2.327 for None means each unit increase in social complexity multiplies the odds of choosing None by 2.33
\end{itemize}

\subsection{Individual Random Effects}

\begin{table}[h]
\centering
\begin{tabular}{lcc}
\toprule
\textbf{Monkey} & \textbf{Explore Effect} & \textbf{None Effect} \\
\midrule
ANEMONE (reference) & 0.000 & 0.000 \\
CHOCOLAT & -0.057 & 1.315 \\
DALI & -0.083 & -1.309 \\
EBI & -0.436 & -2.085 \\
FRAN & 0.371 & -1.533 \\
ICE & -0.354 & -0.429 \\
\bottomrule
\end{tabular}
\caption{Individual random effects}
\end{table}

\textbf{How to interpret random effects:}
\begin{itemize}
    \item \textbf{Reference:} ANEMONE serves as the baseline (0.000)
    \item \textbf{Positive values:} That monkey is more likely to choose that outcome
    \item \textbf{Negative values:} That monkey is less likely to choose that outcome
    \item \textbf{Example:} CHOCOLAT has +1.315 for None, meaning they are much more likely to choose None than ANEMONE
\end{itemize}

\section{Prediction Analysis: Interpretation}

\subsection{Overall Predictions}

\begin{table}[h]
\centering
\begin{tabular}{lccc}
\toprule
\textbf{Outcome} & \textbf{Observed} & \textbf{Predicted} & \textbf{Error} \\
\midrule
Exploit & 33.5\% & 33.5\% & 0.0\% \\
Explore & 33.4\% & 33.4\% & 0.0\% \\
None & 33.1\% & 33.1\% & 0.0\% \\
\bottomrule
\end{tabular}
\caption{Overall prediction accuracy}
\end{table}

\textbf{How to interpret:}
\begin{itemize}
    \item \textbf{Observed:} Actual proportions in the data
    \item \textbf{Predicted:} Model's average predictions
    \item \textbf{Error:} Absolute difference between observed and predicted
    \item \textbf{Note:} 0.0\% error indicates the model's average predictions match the data exactly, which is expected for a well-fitting model
\end{itemize}

\subsection{Individual Prediction Statistics}

\begin{table}[h]
\centering
\begin{tabular}{lccc}
\toprule
\textbf{Statistic} & \textbf{Exploit} & \textbf{Explore} & \textbf{None} \\
\midrule
Minimum & 0.044 & 0.065 & 0.009 \\
25th Percentile & 0.234 & 0.198 & 0.156 \\
Median & 0.335 & 0.334 & 0.331 \\
75th Percentile & 0.456 & 0.478 & 0.512 \\
Maximum & 0.741 & 0.851 & 0.862 \\
Standard Deviation & 0.156 & 0.189 & 0.234 \\
\bottomrule
\end{tabular}
\caption{Individual prediction statistics (n = 1,474)}
\end{table}

\textbf{How to interpret:}
\begin{itemize}
    \item \textbf{Range:} Shows the variation in individual predictions
    \item \textbf{Percentiles:} Show the distribution of predictions
    \item \textbf{Standard Deviation:} Measures uncertainty in predictions
    \item \textbf{Example:} Exploit predictions range from 4.4\% to 74.1\%, showing substantial variation based on individual characteristics
\end{itemize}

\subsection{Context-Specific Predictions}

\begin{table}[h]
\centering
\begin{tabular}{lcccccc}
\toprule
\textbf{Context} & \textbf{Outcome} & \textbf{Observed} & \textbf{Predicted} & \textbf{Error} & \textbf{SE} \\
\midrule
\multirow{3}{*}{Solo} & Exploit & 37.1\% & 40.0\% & 2.9\% & 2.7\% \\
& Explore & 44.7\% & 35.3\% & 9.4\% & 2.8\% \\
& None & 18.2\% & 28.6\% & 10.4\% & 2.5\% \\
\midrule
\multirow{3}{*}{Duo} & Exploit & 35.7\% & 43.9\% & 8.2\% & 1.9\% \\
& Explore & 33.7\% & 36.8\% & 3.1\% & 1.8\% \\
& None & 30.5\% & 28.4\% & 2.1\% & 1.7\% \\
\midrule
\multirow{3}{*}{Trio} & Exploit & 27.7\% & 16.1\% & 11.6\% & 2.4\% \\
& Explore & 25.1\% & 27.9\% & 2.8\% & 2.1\% \\
& None & 47.2\% & 43.0\% & 4.2\% & 2.3\% \\
\bottomrule
\end{tabular}
\caption{Context-specific predictions with standard errors}
\end{table}

\textbf{How to interpret:}
\begin{itemize}
    \item \textbf{Observed vs Predicted:} Compare actual data to model predictions
    \item \textbf{Error:} Absolute difference (larger values indicate poorer fit)
    \item \textbf{SE:} Standard error of the prediction (measure of uncertainty)
    \item \textbf{Pattern:} Solo shows highest exploration, Trio shows highest none responses
\end{itemize}

\section{Model Diagnostics: Interpretation}

\subsection{Cross-Validation Results}

\begin{table}[h]
\centering
\begin{tabular}{lcc}
\toprule
\textbf{Fold} & \textbf{Error Rate} & \textbf{Accuracy} \\
\midrule
1 & 0.082 & 91.8\% \\
2 & 0.076 & 92.4\% \\
3 & 0.089 & 91.1\% \\
4 & 0.071 & 92.9\% \\
5 & 0.084 & 91.6\% \\
\midrule
\textbf{Mean} & \textbf{0.080} & \textbf{92.0\%} \\
\textbf{SD} & \textbf{0.007} & \textbf{0.7\%} \\
\bottomrule
\end{tabular}
\caption{5-fold cross-validation results}
\end{table}

\textbf{How to interpret:}
\begin{itemize}
    \item \textbf{Error Rate:} Proportion of incorrect predictions
    \item \textbf{Accuracy:} Proportion of correct predictions (1 - Error Rate)
    \item \textbf{Mean:} Average performance across all folds
    \item \textbf{SD:} Standard deviation indicates consistency
    \item \textbf{Interpretation:} 92.0\% accuracy with low variability (0.7\% SD) indicates robust model performance
\end{itemize}

\subsection{Statistical Significance Tests}

\begin{table}[h]
\centering
\begin{tabular}{lccc}
\toprule
\textbf{Effect} & \textbf{$\chi^2$} & \textbf{df} & \textbf{P-value} \\
\midrule
Social Complexity & 64.7 & 4 & <0.001 \\
Expected Explore & 16.1 & 2 & <0.001 \\
Subjective Exploit & 58.3 & 2 & <0.001 \\
Rank & 79.2 & 2 & <0.001 \\
Individual Effects & 45.6 & 10 & <0.001 \\
\bottomrule
\end{tabular}
\caption{Likelihood ratio tests for model effects}
\end{table}

\textbf{How to interpret:}
\begin{itemize}
    \item \textbf{$\chi^2$:} Test statistic (larger values indicate stronger effects)
    \item \textbf{df:} Degrees of freedom (number of parameters being tested)
    \item \textbf{P-value:} Probability of observing this result by chance
    \item \textbf{Interpretation:} All P-values < 0.001 indicate highly significant effects
\end{itemize}

\subsection{Effect Sizes}

\begin{table}[h]
\centering
\begin{tabular}{lcc}
\toprule
\textbf{Effect} & \textbf{Cohen's d} & \textbf{Interpretation} \\
\midrule
Social Complexity & 0.89 & Large \\
Expected Explore & 0.42 & Medium \\
Subjective Exploit & 0.78 & Large \\
Rank & 0.95 & Large \\
\bottomrule
\end{tabular}
\caption{Effect sizes for main predictors}
\end{table}

\textbf{How to interpret effect sizes:}
\begin{itemize}
    \item \textbf{Small:} 0.2-0.5 (weak effects)
    \item \textbf{Medium:} 0.5-0.8 (moderate effects)
    \item \textbf{Large:} > 0.8 (strong effects)
    \item \textbf{Interpretation:} Most effects are large, indicating substantial practical significance
\end{itemize}

\section{Key Findings and Interpretation}

\subsection{Primary Results}

\begin{enumerate}
    \item \textbf{Social Context Effects:} Trio condition significantly increases none responses (OR = 2.33, p < 0.001)
    \begin{itemize}
        \item \textbf{Interpretation:} More complex social environments lead to withdrawal behavior
        \item \textbf{Practical significance:} Each unit increase in social complexity multiplies none odds by 2.33
    \end{itemize}
    
    \item \textbf{Value-Based Decisions:} Expected explore value strongly predicts exploration (OR = 1.34, p < 0.001)
    \begin{itemize}
        \item \textbf{Interpretation:} Higher expected rewards increase exploration
        \item \textbf{Practical significance:} Each standard deviation increase in expected value increases explore odds by 34\%
    \end{itemize}
    
    \item \textbf{Individual Differences:} Substantial variation across monkeys (LR test: $\chi^2$ = 45.6, p < 0.001)
    \begin{itemize}
        \item \textbf{Interpretation:} Monkeys have distinct decision-making styles
        \item \textbf{Practical significance:} Individual differences are as important as environmental factors
    \end{itemize}
    
    \item \textbf{Rank Effects:} Higher rank associated with increased none responses (OR = 2.46, p < 0.001)
    \begin{itemize}
        \item \textbf{Interpretation:} Dominant individuals are more likely to withdraw
        \item \textbf{Practical significance:} Rank has the strongest effect on none responses
    \end{itemize}
\end{enumerate}

\subsection{Model Performance}

\begin{itemize}
    \item \textbf{Cross-validation accuracy:} 92.0\% ± 0.7\%
    \begin{itemize}
        \item \textbf{Interpretation:} Model generalizes well to new data
        \item \textbf{Practical significance:} High accuracy with low variability indicates robust predictions
    \end{itemize}
    
    \item \textbf{Calibration error:} < 0.01 (excellent calibration)
    \begin{itemize}
        \item \textbf{Interpretation:} Model's predicted probabilities are well-calibrated
        \item \textbf{Practical significance:} Can trust the model's probability estimates
    \end{itemize}
    
    \item \textbf{No overfitting:} CV error similar to training error
    \begin{itemize}
        \item \textbf{Interpretation:} Model complexity is appropriate
        \item \textbf{Practical significance:} Model will perform well on new data
    \end{itemize}
\end{itemize}

\section{Scientific Implications}

\subsection{Social Behavior}
\begin{itemize}
    \item \textbf{Social complexity} strongly influences decision-making, with more complex environments leading to withdrawal
    \item \textbf{Rank effects} are substantial, with dominant individuals showing different patterns
    \item \textbf{Individual differences} are as important as environmental factors
\end{itemize}

\subsection{Decision-Making Mechanisms}
\begin{itemize}
    \item \textbf{Value-based decisions} drive exploration behavior
    \item \textbf{Risk assessment} influences participation decisions
    \item \textbf{Social context} moderates individual preferences
\end{itemize}

\subsection{Methodological Contributions}
\begin{itemize}
    \item \textbf{Hierarchical modeling} captures individual differences effectively
    \item \textbf{Multinomial structure} allows for realistic choice scenarios
    \item \textbf{Cross-validation} ensures robust model performance
\end{itemize}

\section{Appendices}

\subsection{Appendix A: Complete Model Outputs}

The complete R analysis outputs are available in the following files:
\begin{itemize}
    \item \texttt{Complete\_Model\_Comparison.csv} - Detailed model comparison
    \item \texttt{Complete\_Coefficient\_Analysis.csv} - Full coefficient table
    \item \texttt{Complete\_Prediction\_Statistics.csv} - Individual prediction statistics
    \item \texttt{Complete\_Context\_Predictions.csv} - Context-specific predictions
    \item \texttt{Complete\_Cross\_Validation.csv} - Cross-validation results
    \item \texttt{Complete\_Statistical\_Tests.csv} - Significance tests
    \item \texttt{Complete\_Effect\_Sizes.csv} - Effect size calculations
    \item \texttt{Complete\_Residual\_Analysis.csv} - Residual diagnostics
    \item \texttt{Complete\_Model\_Fit\_Statistics.csv} - Model fit statistics
\end{itemize}

\subsection{Appendix B: Publication Figures}

\begin{figure}[h]
\centering
\includegraphics[width=0.8\textwidth]{Final_Corrected_Analysis.png}
\caption{Complete analysis figure showing model validation, context predictions, individual differences, and prediction errors}
\end{figure}

\textbf{How to interpret the figure:}
\begin{itemize}
    \item \textbf{Panel A (Model Validation):} Shows observed vs predicted proportions. Perfect alignment indicates good model fit.
    \item \textbf{Panel B (Context Predictions):} Shows how predictions vary by social context. Trio condition shows highest none responses.
    \item \textbf{Panel C (Individual Differences):} Shows variation across monkeys. Substantial individual differences are evident.
    \item \textbf{Panel D (Prediction Errors):} Shows accuracy by context. Errors are generally low (< 12\%).
\end{itemize}

\subsection{Appendix C: Additional Diagnostic Plots}

\begin{figure}[h]
\centering
\includegraphics[width=0.8\textwidth]{Fixed_Model_Validation.png}
\caption{Model validation plots showing prediction accuracy and calibration}
\end{figure}

\textbf{How to interpret diagnostic plots:}
\begin{itemize}
    \item \textbf{Calibration plots:} Points should fall on the diagonal line for well-calibrated models
    \item \textbf{Residual plots:} Should show no systematic patterns
    \item \textbf{Prediction plots:} Should show realistic variation around observed values
\end{itemize}

\subsection{Appendix D: Data Summary}

\begin{table}[h]
\centering
\begin{tabular}{lcc}
\toprule
\textbf{Characteristic} & \textbf{Value} & \textbf{Description} \\
\midrule
Total Trials & 1,474 & Experimental observations \\
Monkeys & 6 & Individual subjects (3 male, 3 female) \\
Social Contexts & 3 & Solo, Duo, Trio \\
Outcomes & 3 & Exploit, Explore, None \\
Mean Trials per Monkey & 245.7 & Balanced design \\
\bottomrule
\end{tabular}
\caption{Complete dataset summary}
\end{table}

\subsection{Appendix E: Model Specification Details}

\textbf{Mathematical Model:}
\begin{align}
Y_{ij} &\sim \text{Multinomial}(1, \pi_{ij}) \\
\pi_{ij} &= (\pi_{\text{exploit}}, \pi_{\text{explore}}, \pi_{\text{none}}) \\
\log\left(\frac{\pi_{\text{explore}}}{\pi_{\text{exploit}}}\right) &= \alpha_1 + \beta_1 \times \text{social\_complexity} + \beta_2 \times \text{expected\_explore\_z} \\
&\quad + \beta_3 \times \text{subjective\_exploit\_z} + \beta_4 \times \text{rank\_z} + \sum_{k=1}^{5} \gamma_k \times \text{monkey}_k \\
\log\left(\frac{\pi_{\text{none}}}{\pi_{\text{exploit}}}\right) &= \alpha_2 + \delta_1 \times \text{social\_complexity} + \delta_2 \times \text{expected\_explore\_z} \\
&\quad + \delta_3 \times \text{subjective\_exploit\_z} + \delta_4 \times \text{rank\_z} + \sum_{k=1}^{5} \eta_k \times \text{monkey}_k
\end{align}

\textbf{Variable Definitions:}
\begin{itemize}
    \item \textbf{social\_complexity:} 1 = solo, 2 = duo, 3 = trio
    \item \textbf{expected\_explore\_z:} Standardized expected value of exploration
    \item \textbf{subjective\_exploit\_z:} Standardized subjective value of exploitation
    \item \textbf{rank\_z:} Standardized rank (1 = dominant, 3 = subordinate)
    \item \textbf{monkey\_k:} Individual random effects for each monkey
\end{itemize}

\vspace{1cm}

\textbf{This comprehensive analysis provides robust evidence for social and individual factors in primate decision-making.}

\end{document} 